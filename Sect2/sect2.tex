\section{公式}
本节介绍公式。
一般需要添加宏包:amssymb,amsthm, amsmath。

\subsection{行内公式}
这里有一个行内公式:$E=mc^2$,行内公式处于两个``\$"之间。

$\alpha\beta\Theta\Xi\zeta\Re\Im\wp\infty\partial$等等诸如此类的符号,如果你不太清楚写法可以打开View-->TeX GUI Symbols...然后用鼠标戳进来,分数写法:$\frac{a}{b}$。

这里是上标和下标的写法:
\begin{itemize}
  \item 上标:$a^2$,$a^{x+y}$,$a^{b^{c^d}}$,$^aX^a$。

  当上标或下标字符超过一时要用大括号括起来
  \item 下标:$a_2$, $a_{x+y}$,$a_{b_{c_d}}$
\end{itemize}

\subsection{行间公式}
一般行间公式的写法:
\begin{equation}
  a+b=c
  \label{abc}
\end{equation}

这样一个公式写出来就会自动编号,\text{\label{abc}}给这个公式起了个名字,在其他的地方就可以应用它,比如: 式\ref{abc}中讲到$c$是$a$和$b$的和。
\subsubsection{矩阵}
\begin{equation}
A=
 \left[
  \begin{array}{ccc}% c表示元素居中,也可以用l,r分别表示左右对其
    1 & 2 & 3\\
    3 & 4 & 5\\
    5 & 6 & 7\\
  \end{array}
  \right]
\end{equation}

\subsubsection{公式换行对齐}
\begin{equation}
  \begin{aligned}
  f&=a+b+c+d\\% &设置对齐点
   &=e+f+h+j\\
   &=+z+x+c+v\\
  \end{aligned}
\end{equation} 

如果有的时候你不希望显示某个公式的编号只需要在equation后面加一个``*'':
\begin{equation*}
  \int_a^bf(x)\mathrm{d}x=F(x)
\end{equation*}