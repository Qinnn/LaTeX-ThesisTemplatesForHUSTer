\section{插图}
本节介绍插图。

一般用到的宏包有:graphicx,subfigure等。

请将需要插入的图片放置到相应章节的文件夹中。

最常用的插图如下:

\begin{figure}[htbp]
 \centering
        \includegraphics[width=0.5\columnwidth]{hello.png}
        \caption{
                \label{hello}
                这里是hello
        }
\end{figure}

centering控制图形居中,htbp 表示图形首选放在当前插图的位置,然后是页面头部,然后是页面尾部。width控制插图宽度(默认图形约束长宽比),后面接要插入的图片名,caption给出图形下面的描述,label为图形建立索引,在别的地方可以用ref来引用。例如:图\ref{hello}里面是hello。

\subsection{并排插多图}

一般使用minipage来达到这个目的具体请看本页tex源码:

\begin{figure}[htbp]
\begin{minipage}{0.5\linewidth}
\centering
\includegraphics[width=0.5\textwidth]{hello.png}
\caption{caption1}
\label{hello1}
\end{minipage}%
\begin{minipage}{0.5\linewidth}
\centering
\includegraphics[width=0.5\textwidth]{hello.png}
\caption{caption2}
\label{hello2}
\end{minipage}
\end{figure}

\subsection{一幅图多个子图}

一般使用subfigure来达到这个目的具体请看本页tex源码:

\begin{figure}[htb]
  \centering
  \subfigure[子图单独的caption]{
    \label{hello3} %% label for first subfigure
    \includegraphics[width=0.3\textwidth]{hello.png}}
  %\vspace{1in}
  \subfigure[留白可取消]{
    \label{hello4} %% label for second subfigure
    \includegraphics[width=0.3\textwidth]{hello.png}}
    \caption{总的caption}
  \label{hello5} %% label for entire figure
\end{figure} 

更多的插图需求请访问\url{http://www.ctex.org/documents/latex/graphics/}.